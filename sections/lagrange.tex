\setcounter{section}{3}
\section{Lagrange Mechanics}
\setcounter{section}{8}
\setcounter{subsection}{1}
\subsection{Lagrange Mechanics} %% ------------------------- 8 . 2 -------------

\textbf{The lagrangian}. Define a set of generalized coordinates \(\bm{q}\). Let
 \(T(\bm{q},\dot{\bm{q}},t)\) be the kinetic energy and 
\(U(\bm{q},\bm{\dot{q}},t)\) the potential energy (Sometimes \(V\)). Then the lagrangian is
defined to be
\begin{align*}
    L(\bm{q},\dot{\bm{q}},t) = T(\bm{q},\dot{\bm{q}},t)-U(\bm{q})
\end{align*}
The equation of motion is
\begin{align*}
    \frac{d}{dt}\left(\frac{\partial L}{\partial \dot{q_i}}\right) - \frac{\partial L}{\partial q_i} = \tau_i \\
    \frac{d}{dt}\left(\nabla_{\dot{\bm{q}}}L\right) - \nabla_{\bm{q}}L = \bm{\tau}
\end{align*}
Where \(\tau_i\) is a generalized actuator force
\begin{align*}
    \tau_i = \sum_{k=1}^{N}\frac{\partial \bm{r_k}}{\partial q_i} \cdot \bm{F}_k
\end{align*}
\(\bm{r}_k(\bm{q})\) is the position of the point of application of the force \(\bm{F}_k\).
In general \(\tau_i\) is a force or a torque.
\newline

\textbf{Constrained Lagrange}. Having the constraints
\begin{align*}
    \bm{c}(\bm{q}) = \bm{0}
\end{align*}
The system can be described by
\begin{align*}
    & L(\bm{q},\dot{\bm{q}},t) = T(\bm{q},\dot{\bm{q}},t)-U(\bm{q})  - \bm{z}^T\bm{c}(\bm{q}) \\
    & \frac{d}{dt}\left(\nabla_{\dot{\bm{q}}}L\right) - \nabla_{\bm{q}}L = \bm{\tau} \\
    & \bm{c}(\bm{q}) = \bm{0}
\end{align*}

\textbf{Baumgarte stabilization}. Instead of imposing
\[\ddot{\bm{c}}(\bm{q}) =\bm{0}\]
Impose
\[\ddot{\bm{c}}+2\alpha\dot{\bm{c}}+\alpha^2\bm{c} = \bm{0}\]
As to reduce drifts in the constraints resulting from 
\[\ddot{\bm{c}}=\bm{0}\]
not begin satisfied exactly when doing numeric computations.


\setcounter{section}{4}
\section{Differential Algebraic Equations}
\setcounter{section}{14}
\setcounter{subsection}{1}
\subsection{Preliminaries} %% ------------------------------ 14 . 2 ------------

\setcounter{section}{5}
\setcounter{subsection}{0}
\subsection{Differential Algebraic Equations} %% ----------- 5 -----------------

\textbf{Definition of DEA}. The differential equation defined by
\begin{align*}
    \bm{F}(\dot{\bm{x}},\bm{x},\bm{u},t) = \bm{0}
\end{align*}
Es a DAE if
\begin{align*}
    \frac{\partial \bm{F}}{\partial \dot{\bm{x}}} 
\end{align*}
is rank deficient
\newline

\textbf{Fully-explicit DAE}
\begin{align*}
    & \bm{F}(\dot{\bm{x}},\bm{x},\bm{z},\bm{u}) = \bm{0} \\
    & \det\left|\frac{\partial \bm{F}}{\partial \dot{\bm{x}}}\right| = 0
\end{align*}
Can be rewritten as
\begin{align*}
    & \dot{\bm{x}} = \bm{v} \\
    & \bm{0} = \bm{F}(\bm{v},\bm{x},\bm{z},\bm{u}) \\
\end{align*}

\textbf{Semi-explicit DAE}
\begin{align*}
    & \dot{\bm{x}} = \bm{f}(\bm{x},\bm{z},\bm{u}) \\
    & \bm{0} = \bm{g}(\bm{x},\bm{z},\bm{u}) \\
\end{align*}
This can be rewritten as
\begin{align*}
    & \bm{F}(\dot{\bm{x}},\bm{x},\bm{z},\bm{u}) = \begin{pmatrix}
        \dot{\bm{x}} - \bm{f}(\bm{x},\bm{z},\bm{u}) \\ \bm{g}(\bm{x},\bm{z},\bm{u})
    \end{pmatrix} = \bm{0}\\
\end{align*}

\textbf{Tikhonov Theorem}. Consider the ordinary differential equation
\begin{align*}
    \dot{\bm{x}} = \bm{f}(\bm{x},\bm{z}) \\
    \varepsilon \dot{\bm{z}} = \bm{g}(\bm{x},\bm{z})
\end{align*}
If \begin{itemize}
    \item dynamics of \(\dot{\bm{z}} = \bm{g}(\bm{x},\bm{z})\) stable \(\forall \bm{x}\)
    \item \(\frac{\partial \bm{g}}{\partial \bm{z}}\) is full rank
\end{itemize}
then
\begin{align*}
    \lim_{\epsilon \rightarrow 0} \bm{x}(t) = \bm{x}_0(t) \\
    \lim_{\epsilon \rightarrow 0} \bm{z}(t) = \bm{z}_0(t)
\end{align*}
where \(\bm{z}_0(t)\) and \(\bm{x}_0(t)\) is the solution to the ODE above modified to a DAE where \(\varepsilon = 0\)
\newline

\textbf{Theorem: Solvability of DAE}. A fully implicit DAE with smooth
\begin{align*}
    \bm{F}(\dot{\bm{x}},\bm{x},\bm{z},\bm{u}) = \bm{0}
\end{align*}
Can be readily solved (solved for \(\dot{\bm{x}}\) and \(\bm{z}\)) if
\begin{align*}
    \begin{pmatrix}\frac{\partial \bm{F}}{\partial \dot{\bm{x}}} & \frac{\partial \bm{F}}{\partial \bm{z}} \end{pmatrix}
\end{align*}
is full rank on all trajectories \(\dot{\bm{x}}\), \(\bm{z}\), \(\bm{x}\) and \(\bm{u}\). Note that all \textbf{Index 1} DAEs fullfil these requirements. The theorem implies that
\begin{align*}
    \dot{\bm{x}} = \bm{f}(\bm{x},\bm{z},\bm{u}) \\
    \bm{0} = \bm{g}(\bm{x},\bm{z},\bm{u})
\end{align*}
with smooth \(\bm{f}\) can be solved if
\begin{align*}
    \frac{\partial \bm{g}}{\partial \bm{z}}
\end{align*}
is full rank on all trajectories \(\bm{z}\), \(\bm{x}\) and \(\bm{u}\)
\newline

\textbf{Definition: Differential index of a DAE} is the number of times the differentiation operator \(\frac{d}{dt}\)
must be applied to the equations in order to convert the DAE into an ODE.

\setcounter{section}{7}
\section{Mathematics}
\textbf{Inverse of \(2\times2\) Matrix}
\begin{align*}
    \begin{pmatrix}
        a & b \\ c & d 
        \end{pmatrix}^{-1} = 
    \frac{1}{ad-bc}
    \begin{pmatrix}
        d & -b \\ -c & a 
    \end{pmatrix}
\end{align*}

\textbf{Multivariable derivative rules}
\begin{align*}
    \frac{\partial}{\partial\bm{x}} \bm{a}^T\bm{x} &=  \bm{a}^T &
    \nabla_{\bm{x}}\bm{a}^T\bm{x} &=  \bm{a} \\
    \frac{\partial}{\partial\bm{x}} \bm{Ax} &=  \bm{A} &
    \nabla_{\bm{x}}\bm{Ax} &= \bm{A}^T  \\
    \frac{\partial}{\partial\bm{x}} \bm{x}^T\bm{A} &=  \bm{A}^T &
    \nabla_{\bm{x}}\bm{x}^T\bm{A} &= \bm{A} \\
\end{align*}
Second order terms
\begin{align*}
    \frac{\partial}{\partial\bm{x}} \frac{1}{2}\bm{x}^T\bm{Ax} &=  \frac{1}{2}\bm{x}^T(\bm{A}^T+\bm{A}) \\
    \nabla_{\bm{x}} \frac{1}{2}\bm{x}^T\bm{Ax} &=  \frac{1}{2}(\bm{A}^T+\bm{A})\bm{x} \\
\end{align*}

\textbf{Multivariable Chain rule}
\begin{align*}
    \frac{\partial\bm{f}(\bm{g}(\bm{x}))}{\partial\bm{x}} &= \frac{\partial\bm{f}}{\partial\bm{g}}
    \frac{\partial\bm{g}}{\partial\bm{x}} \\
    \frac{\partial\bm{f}(\bm{g}(\bm{x}),\bm{h}(\bm{x}))}{\partial\bm{x}} &= \frac{\partial\bm{f}}{\partial\bm{g}}
     \frac{\partial\bm{g}}{\partial\bm{x}} + \frac{\partial\bm{f}}{\partial\bm{h}}\frac{\partial\bm{h}}{\partial\bm{x}}
\end{align*}

\textbf{Some derivatives}
\begin{align*}
    &\frac{d}{dt}\sinh(t) &&= \cosh(t) \\
    &\frac{d}{dt}\cosh(t) &&= \sinh(t) \\
    &\frac{d}{dt}\tanh(t) = \frac{d}{dt}\frac{\sinh(t)}{\cosh(t)} &&= \frac{1}{\cosh^2t} = 1-\tanh^2(x) \\
    &\frac{d}{dt}\arctan(t) &&= \frac{1}{1+t^2}
\end{align*}

\textbf{Taylor's theorem} \newline
Let \(k\geq1\) be an integer and let the function \(f:\RR \rightarrow \RR\) be \(k+1\) times differentiable at the point \(a\in\RR\). Then
\begin{align*}
    f(x) &= \sum_{n=0}^{k}\frac{1}{n!}f^{(n)}(a)(x-a)^n \\
     &+ \frac{1}{(1+k)!}f^{(k+1)}(\xi)(x-a)^{k+1}
\end{align*}
for some \(\xi\in[x,a]\) \newline


\textbf{Common Taylor expansions}
\begin{align*}
    e^x &= \sum_{n=0}^{\infty}\frac{x^n}{n!} \\
    \sin(x) &= \sum_{n=1}^{\infty}(-1)^n\frac{x^{2n+1}}{(2n+1)!} \\
    \cos(x) &= \sum_{n=0}^{\infty}(-1)^n\frac{x^{2n}}{(2n)!}
\end{align*}

\textbf{Integration in polar coordinates}
\begin{align*}
    x & = r \cos \theta \\
    y & = r \sin \theta \\
    dA & = r\,dr\,d\theta
\end{align*}

\textbf{Integration in spherical coordinates}
\begin{align*}
    x & = \rho \sin \phi \cos \theta \\
    y & = \rho \sin \phi \sin \theta \\
    z & = \rho \cos \phi \\
    dV &= \rho^2 \sin \phi \, d\rho \, d\phi \, d\theta
\end{align*}

\textbf{trigonometric identities}
\begin{align*}
    i \sin \theta &= \sinh {i\theta} \\
    \cos \theta &= \cosh {i\theta} \\
    \cosh x & = (e^x+e^{-x})/2 \\
    \sinh x & = (e^x-e^{-x})/2 \\
    \sin^2 x & = (1-\cos{2x})/2 \\
    \cos^2 x & = (1+\cos{2x})/2
\end{align*}

\textbf{Implicit function theorem in} \(\RR^{n+1}\) \newline
Let \(\mathcal{D}\) be an open subset of \(\RR^{n+1}\) and let \(f:\mathcal{D}\rightarrow\RR\)
be a function of the variables \(\bm{x}\) and \(y\) with continuous partial derivatives. Assume
the point \((\bm{a},b)\in\mathcal{D}\) fulfills 
\[f(\bm{a},b) = 0 \textrm{ and } \frac{\partial f}{\partial y}(\bm{a},b) \neq 0\]
then there exists a differentiable function \(g:A\rightarrow \RR\) where
\[A=\{\bm{x}\in\RR^n:||\bm{x}-\bm{a}||<\rho\}\]
for some \(\rho > 0\). \(g\) satisfies
\[g(\bm{a})=b \textrm{ and } f(\bm{x},g(\bm{x}))=0\]
The derivative of \(g\) is
\[\frac{\partial g}{\partial x_i}(\bm{x}) = -\frac{\frac{\partial f}{\partial x_i}(\bm{x},g(\bm{x}))}{\frac{\partial f}{\partial y}(\bm{x},g(\bm{x}))}\]


\textbf{Miscellaneous}
\begin{align*}
    \bm{x}^\times\bm{x}^\times = \bm{x}\bm{x}^T - \bm{x}^T\bm{x}\mathbb{I}
\end{align*}

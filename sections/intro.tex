\section{Intro}

\setcounter{subsection}{1}
\subsection{State space methods}
\subsubsection{State space models} %% ---------------------- 1 . 2 . 1 ---------
\textbf{State Space Model} is on the form
\begin{align*}
    \dot{\bm{x}} = \bm{f}(\bm{x},\bm{u},t)
\end{align*}

\textbf{Linear time invariant system}
\begin{align*}
    \dot{\bm{x}} &= \bm{Ax}+\bm{Bu} \\
    \bm{y} &= \bm{Cx + Du} \\
    \bm{y} &= \bm{C}e^{\bm{A}t}\bm{x}(0) + \int_0^t\bm{C}e^{\bm{A}(t-\tau)}\bm{B}\bm{u}(\tau)\,d\tau+\bm{D}\bm{u}(t)
\end{align*}

\subsubsection{Second order models of mechanical systems} %% 1 . 2 . 2 ---------
Second order models on the form
\begin{align*}
    \bm{M}(\bm{q})\bm{\ddot{q}} + \bm{f}(\bm{q},\bm{\dot{q}}) = \bm{u}
\end{align*}
Can be written as
\begin{align*}
    \dot{\begin{pmatrix}
        \bm{q} \\ \bm{\dot{q}}
    \end{pmatrix}} = \begin{pmatrix}
        \bm{\dot{q}} \\  \bm{M}^{-1}(\bm{q})(-\bm{f}(\bm{q},\bm{\dot{q}})+\bm{u})
    \end{pmatrix}
\end{align*}
\subsubsection{Linearization of state space models} %% ----- 1 . 2 . 3 ---------

\textbf{Linearization of time varying systems}
\begin{align*}
    \dot{\bm{x}} &= \bm{f}(\bm{x},\bm{u},t) \\
    \bm{y} &= \bm{h}(\bm{x},\bm{u},t)
\end{align*}
Find two functions \(\bm{x}_0\) and \(\bm{u}_0\) being solutions to the system
\begin{align*}
    \dot{\bm{x}}_0 = \bm{f}(\bm{x}_0(t),\bm{u}_0(t),t)
\end{align*}
Define perturbations
\begin{align*}
    \bm{x}(t) &= \bm{x}_0(t) + \bm{\Delta x}(t) \\
    \bm{u}(t) &= \bm{u}_0(t) + \bm{\Delta u}(t) \\
    \bm{y}(t) &= \bm{y}_0(t) + \bm{\Delta y}(t)
\end{align*}
Let \(\mathcal{C} = \{\bm{x}_0(t),\bm{u}_0(t)\}\). The linearized system is
\begin{align*}
    \bm{\Delta \dot{x}} &\approx \frac{\partial\bm{f}}{\partial\bm{x}}\Big|_{\mathcal{C}} \bm{\Delta x} +
        \frac{\partial\bm{f}}{\partial\bm{u}}\Big|_{\mathcal{C}}\bm{\Delta u} \\
    \bm{\Delta \dot{y}} &\approx \frac{\partial\bm{h}}{\partial\bm{x}}\Big|_{\mathcal{C}} \bm{\Delta x} +
        \frac{\partial\bm{h}}{\partial\bm{u}}\Big|_{\mathcal{C}}\bm{\Delta u} \\
\end{align*}

\setcounter{subsection}{4}
\subsection{ODE's} %% From the compendium
General formulation
\begin{align*}
    \varphi(y^{(m)},\cdots,y,u^{(m-1)},\cdots,u) = 0
\end{align*}

\textbf{Lipschitz continuous}.A function \[f:\RR \rightarrow \RR\] is said to be \textit{Lipschitz continuous} if
\begin{align*}
    &\exists c > 0 \in \RR & \textrm{s.t.}\\ &||f(x)-f(y)||<c||x-y|| & \forall x, y \in \RR
\end{align*}

\textbf{Theorem: existence of unique solution}. Consider the ODE
\[\dot{x}=f(x)\]
If \(f\) is Lipschitz continuous then \(x(t)\) exists and is unique for all \(t\)
\newline

\textbf{Mean Value Theorem} 
suppose \(f\) is continuous on \([a,b]\) and differentiable on \((a,b)\) then
\begin{align*}
    \exists c\in(a,b) \textrm{ s.t. } f'(c)=\frac{f(b)-f(a)}{b-a}
\end{align*}
This can be used to show that if \(f\) is continuous and differentiable everywhere it is also Lipschitz.
\newline

\textbf{Theorem 2: existence of unique solution}. Consider the ODE
\[\dot{x}=f(x)\]
if \(f\) is continuously differentiable (\(\frac{\partial f}{\partial x}\) exists and is continuous), then the solution to the ODE exists and is unique on some time interval.

